\documentclass{beamer}
\usepackage[italian]{babel} 
 \usetheme{CambridgeUS} 
\usepackage[T1]{fontenc}
\setbeamertemplate{blocks}[rounded][shadow=true]

\title{Algoritmi per il calcolo parallelo}
\author{...........Author..............}
\date{2021}
 
\subtitle{Equazione dei flussi a superficie libera su griglie strutturate}

\begin{document}
	\frame{\titlepage}
	
	\begin{frame}
		\frametitle{Introduzione}
		Lo scopo di questo progetto è la simulazione di una situazione fluidodinamica: correnti a superficie libera su griglie strutturate.
		\medskip
	
		Così facendo siamo in grado di osservare non solo la profondità dell'acqua in ogni punto ma anche la conformazione fisica di questa all'interno dell'area di studio. 
		
		\medskip
		Per poter così ottimizzare le tempistiche di calcolo, la simulazione potrà essere messa in esecuzione con un qualsiasi numero di processori.
		
		\smallskip
		
		Infatti, lo dimostreremo nel momento in cui andremo a guardare i vari tempi di speed-up effettuati con un numero di processori differenti.
	
	\end{frame}

	\begin{frame}
		\frametitle{Condizioni}

		Il nostro dominio di calcolo è: $\Omega$ = $ [ -0.5; 0.5 ] x [ -0.5; 0.5 ]	$, con la condizione iniziale della nostra pressione $\eta$ =  1 + $e^{ -\frac{1}{2* s^{2}} * (x^{2} + y^{2})} $ e \\ \emph{s} = 0.1. che è il tempo, con intervallo [0, 0.1]\\
		
		
		Il dominio di calcolo è discretizzato per mezzo di una griglia sfalsata composta da un numero totale, N e M, di celle equamente risparmiate con dimensione della maglia ($\Delta$x ).
		
		
		Per questo progetto, abbiamo tralasciato i termini convettivi, il che significa : 
			\begin{center}
			
			$ \emph{Fu}_{i + \frac{1}{2} j}^{n} = u_{i+\frac{1}{2},j}^{n}	$ e $\emph{Fv}_{i,j + \frac{1}{2}} ^{n}	= u_{i, j +\frac{1}{2},j}^{n}	$
			
			\end{center}	
	\end{frame}


	\begin{frame}
		\frametitle{Condizioni}
			\begin{itemize}
				\item Le velocità discrete u($i+\frac{1}{2} ; j $) e v($ i ; j+\frac{1}{2}$) sono definite alle interfacce delle celle, mentre la pressione discreta ($\eta$) è data ai baricentri delle celle.\\
				\item $\Delta$t rappresenta il passo temporale.
				\smallskip
				
				\item Ricorderemo brevemente le equazioni di governo per i flussi a superficie libera e introdurremo un metodo numerico adatto per risolverle. Lo schema è basato su una discretizzazione temporale semi-implicit e un'approssimazione spaziale a differenza finita.
				\item Lasciamo che lo spazio fisico sia descritto dal vettore posizione x=(x,y,z), mentre gli componenti di velocità in direzione x, y e z siano indicati con u(x, y, z, t), v(x, y, z, t), rispettivamente con t inizia il tempo.
			\end{itemize}
	\end{frame}
	
	\begin{frame}
		\frametitle{Condizioni}
			La profondità totale dell'acqua all'interfaccia è calcolata come: 
	
		\begin{center}
			$ H_{i+\frac{1}{2};j}^{n} $ = $  \max { 0, h_{i + \frac{1}{2}; j}^{n} + \eta_{i;j}^{n} ; h_{i+\frac{1}{2}; j}^{n} + \eta_{i+1;j)^{n} } } $
			
			$ H_{i; j+\frac{1}{2}}^{n} $ = $ \max { 0 , h_{i; j+\frac{1}{2}}^{n} + \eta_{i;j}^{n}; h_{i; j+\frac{1}{2}}^{n}+ \eta_{i; j+1)} } $
		\end{center}
	
			
	\end{frame}
	
	\begin{frame}
		\frametitle{Costruzione dell'equazione dei flussi a superficie libera su griglie strutturate }
			
			Abbiamo un sistema di tre equazioni: due per le velocità del fluido e uno per la pressione di esso, cioè la sua altezza. \\
			Se ipotizziamo di avere un fondo piatto, la nostra \emph{H} è la pressione del nostro fluido.\\
			Sostanzialmente, la pressione sarebbe l'altezza che si ha del fluido in un determinato punto.\\

	\end{frame}

	\begin{frame}
		Il sistema è il seguente:\\
		\begin{center}
	
			$ u_{i + \frac{1}{2} ; j}^{n+1} = Fu_{i + \frac{1}{2}; j} ^{n}  - g*\frac{\Delta t}{\Delta x} *( \eta_{i+1; j} ^{n+1} -\eta_{i; j} ^{n+1}) $
			
			$ v_{i ; j + \frac{1}{2}}^{n+1} = Fv_{i;j  + \frac{1}{2}} ^{n}  - g*\frac{\Delta t}{\Delta y} *( \eta_{i; j+1} ^{n+1} -\eta_{i; j} ^{n+1}) $
				
				$ \eta_{i ; j}^{n+1} = \eta_{i ;j} ^{n}  - g*\frac{\Delta t}{\Delta x} *( H_{i+ \frac{1}{2} ; j}^n u_{i+\frac{1}{2};j}^{n+1} - H_{i-\frac{1}{2};j}^{n} u_{i-\frac{1}{2};j}^{n}) $ \\
				$\frac{\Delta t}{\Delta y} *( H_{i; j+ \frac{1}{2}}^{n} v_{i; j+\frac{1}{2}}^{n+1} - H_{i;j-\frac{1}{2}}^{n} v_{i;j-\frac{1}{2}}^{n}) $ 
		\end{center}

	\end{frame}

\begin{frame}
	Dal sistema sopra, possiamo ricavare l'equazione che stiamo cercando, tramite il metodo implicito. 
	L'equazione che ricaviamo, tramite le varie sostituzioni è:
	
	$ \eta_{i;j}^{n} - \frac{\Delta t}{\Delta x}Hx_{i+\frac{1}{2};j}^{n} Fu_{i+\frac{1}{2};j} + \frac{\Delta t^{2}}{\Delta x^{2}}Hx_{i-\frac{1}{2};j}^{n} Fu_{i-\frac{1}{2};j} -\frac{\Delta t^{2}}{\Delta y^{2}}Hy_{i;j+\frac{1}{2}}^{n} Fv_{i;j+\frac{1}{2}}  + \frac{\Delta t^{2}}{\Delta y^{2}}Hy_{i;j-\frac{1}{2}}^{n} Fv_{i;j-\frac{1}{2}} = -g\frac{\Delta t^{2}}{\Delta x^{2}}Hy_{i;j-\frac{1}{2}}^{n} \eta_{i;j-1} -g\frac{\Delta t^{2}}{\Delta x^{2}}Hx_{i;j-\frac{1}{2}}^{n}  \eta_{i-1;j} + (1 + g(\frac{\Delta t^{2}}{\Delta x^{2}}Hx_{i+\frac{1}{2}; j}^{n} +\frac{\Delta t^{2}}{\Delta x^{2}}Hx_{i-\frac{1}{2}; j}^{n} + \frac{\Delta t^{2}}{\Delta y^{2}}Hy_{i;j+\frac{1}{2}}^{n} + \frac{\Delta t^{2}}{\Delta y^{2}}Hy_{i;j-\frac{1}{2}}^{n} ) )\eta_{i;j}^{n+1} - g\frac{\Delta t^{2}}{\Delta x^{2}}Hx_{i+;\frac{1}{2} ; j}^{n}  \eta_{i+1;j}^{n+1} - g\frac{\Delta t^{2}}{\Delta y^{2}}Hy_{i+;\frac{1}{2} ; j}^{n}  \eta_{i;j+1}^{n+1}	$
\end{frame}

	\begin{frame}
	\frametitle{Condizioni}
	\begin{itemize}
		\item Le velocità discrete u($i+\frac{1}{2} ; j $) e v($ i ; j+\frac{1}{2}$) sono definite alle interfacce delle celle, mentre la pressione discreta ($\eta$) è data ai baricentri delle celle.\\
		\item $\Delta$t rappresenta il passo temporale.
		\smallskip
		
		\item Ricorderemo brevemente le equazioni di governo per i flussi a superficie libera e introdurremo un metodo numerico adatto per risolverle. Lo schema è basato su una discretizzazione temporale semi-implicit e un'approssimazione spaziale a differenza finita.
		\item Lasciamo che lo spazio fisico sia descritto dal vettore posizione x=(x,y,z), mentre gli componenti di velocità in direzione x, y e z siano indicati con u(x, y, z, t), v(x, y, z, t), rispettivamente con t inizia il tempo.
	\end{itemize}
\end{frame}

\begin{frame}
	\frametitle{Idea generale}
 	Definiamo delle matrici, es. 120x120 definita in 3 parti, dove le \textcolor{blue}{\textbf{I}} le lascio come sono, mentre la diviso solo sulla base di \textcolor{blue}{\textbf{J}},
 	quindi c'è una divisione orizzontale.\\
 	Le matrici vengono definite inoltre da \textcolor{blue}{\emph{istart -1}} a \textcolor{blue}{\emph{jend +1}}. Questo lo si fa  perché prima di fare determinati conti, quando so quale valore utilizzo in precedenza e quello successivo,nel caso la matrice l'ho suddivisa in più processori, quello che faccio è riservarmi una colonna in più da ogni parte (prendere per ogni processore la sua colonna più a sinistra o a destra calcolata), quindi prendere da \textcolor{blue}{\emph{istart}} a \textcolor{blue}{\emph{iend}}, passarla a quella vicino che se la salva in \textcolor{blue}{\emph{istart - 1}} e \textcolor{blue}{\emph{iend + 1}}
 	Facendo così ottengo i miei valori, in aggiunta ho l'ultima colonna che diventa la mia prima riga e viceversa; li prendo e li utilizzo per effettuare il conto che sto facendo.\\
 	Facendo così possiamo ottenere le informazioni da parte degli altri processori. 
	
\end{frame}
\begin{frame}
	\frametitle{Idea generale}

	
	
\end{frame}
\end{document}